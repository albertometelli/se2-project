
\subsection{Purpose}

The purpose of the RASD \emph{(Requirements Analysis and Specification
Document}) is to give a detailed description, analysis and specification
of the requirements for the \emph{myTaxiService} software. This document
will explain the \emph{goals} coming from stakeholders' expectations,
the characteristics of the application domain and the \emph{assumptions}
made to solve ambiguity and incompleteness. Starting from goals and
domain properties, \emph{requirements} will be formulated according
to a specific systematic methodology and then specified using both
informal and formal notations. However this document should not be
considered the final draft for the software specifications since in
the following phases several fixing may be necessary. 

The main objective of the RASD is to achieve good understanding among
\emph{analysts, developers, testers} and \emph{customers}, in particular
explaining both the application domain and the system to-be; it is
also aimed to be a solid base for project planning, software evaluation
and possible future maintenance activities. Therefore this document
is primarily intended to be proposed to the stakeholders for their
approval, to the analysts and programmers for the development of the
project and to the testing team the validation of the first version
of the software.


\subsection{Present system}

At the moment taxi service is entirely managed by phone calls. A passenger
who wants to request or reserve a taxi has to contact the call center.
In case of request, the call center operator moves the call to the
first available taxi, otherwise, in case of reservation, a taxi is
booked for the specific date, time and address provided by the passenger.
No available taxi queues management is implemented at the moment. 


\subsection{Scope}

The \emph{myTaxyService} is an application intended to optimize taxi
service in a large city, making the access to service simpler for
the passengers and ensuring a fair management of the taxi queues. 

Passengers will be able to request a taxi either through a web application
or a mobile app; of course the ``traditional'' ways to call for
a taxi, like a phone call or stopping the taxi along the road, will
be still available and integrated into the system to-be. The software
will make the procedure of calling a taxi simpler (by using GPS information
passenger doesn't need to know the address if the taxi is needed for
the current position) and more usable (passenger will be provided
with information about the waiting time). Moreover, by means of the
application, the passenger can reserve a taxi for a certain date and
time, specifying the origin and the destination of the ride.

Taxi drivers will use a mobile app to inform the system about their
availability and to confirm that they are going to take care of a
call (or to reject it for any reason). The software will make the
taxi management more efficient: the system will be able to identify
the position of each taxi by using GPS; the city will be divided in
virtual zones and a suitable distribution of the taxi among the zones
will automatically be computed.


\subsection{Definitions, acronyms, abbreviations }

In this paragraph all the terms, acronyms and abbreviations used in
the following sections are listed.


\subsubsection{Definitions}
\begin{itemize}
\item \emph{Request}: the action performed by the passenger of calling a
taxi for the current position.
\item \emph{Confirmed request}: a request that has been accepted by a taxi
driver.
\item \emph{Reservation}: the action performed by the passenger of booking
a taxi for a specific address and specific date and time.
\item \emph{Waiting time}: an estimation of the time required to taxi driver
to get to passenger's position.
\item \emph{Taxi code}: a unique alphanumerical identifier of the taxi.
\item \emph{Available taxi queues}: data structures used to store the references
of the available taxis, also used to select the taxis to which forward
a request.
\item \emph{Automatic geolocalization}: a system that provides the geographic
coordinates of the user. For this document it can be either a GPS
system or browser geolocalization.
\item \emph{Passengers' application}: the applications used by passengers
to access to TS system. For this document it can be either PMA or
PWA (see 1.4.2).
\item \emph{Login credentials}: username and password.
\item \textit{Notification}: communication from TS to taxi driver to move
to a specific zone.
\end{itemize}

\subsubsection{Acronyms}
\begin{itemize}
\item TS: myTaxiService.
\item PMA: Passenger mobile application.
\item PWA: Passenger web application.
\item TMA: Taxi driver mobile application.
\item QMS: Queue management system.
\end{itemize}

\subsubsection{Abbreviations}
\begin{itemize}
\item {[}Gn{]} n-th goal.
\item {[}Dn{]} n-th domain assumption.
\item {[}Rn.m{]} m-th requirement related to goal {[}Gn{]}.
\end{itemize}

\subsection{Actors}

In this paragraph a brief description of the various actors affected
by myTaxiService system is provided.
\begin{itemize}
\item \emph{Passenger}: a person that interacts with myTaxiService to request
or reserve a taxi. The interaction with the system may occur by means
of either PMA (mobile passenger) or PWA (web passenger). Each passenger
can be either a registered passenger or an unregistered passenger.
\item \emph{Registered passenger}: specific case of passenger that has already
registered to the system. He/She can login, request, reserve a taxi
and also visualize and modify the previous reservations.
\item \emph{Unregistered passenger}: specific case of passenger that hasn't
registered to the system. He/She can only request a taxi.
\item \emph{Taxi driver}: a person that drives a taxi and is associated
with myTaxiService. He/She interacts with the system confirming or
rejecting requests and informing the system about his/her availability
by means of TMA.
\item \emph{Call center operator}: a person working at the call center that
interacts with the system inserting taxi requests coming from phone
calls. 
\end{itemize}

\subsection{Requirement engineering approach}

In order to ensure a sound and complete requirement engineering activity,
we decided to follow a systematic technique for requirements formulation
proposed by Jackson and Zave. This approach is based on the distinction
between the \emph{machine}, the portion of the system to be developed
(myTaxiService in our case), and the \emph{world}, the portion of
the real world interacting with the machine. Machine and world are
tipically non disjoint, some phenomena may affect both of them, they
are known as shared phenomena. From this viewpoint, requirement engineering
consists in identifying phenomena shared between world and machine,
according to a set of \textbf{goals }(which express the desired behavior
of world phenomena, shared or not) and a set of \textbf{domain assumptions}
(assertions supposed to be always valid in the world). Formally a
set of \textbf{requirements }is complete if together with domain assumption
it ensures the goals.


\subsubsection{Goals}

Starting from the available documentation, integrated with some interviews
with the stakeholders, the following minimal goals has been identified.
\begin{lyxlist}{00.00.0000}
\item [{{[}G1{]}}] Allow a passenger to request a taxi for its current
position without registration.
\item [{{[}G2{]}}] Allow the passenger to visualize the waiting time and
the code of the incoming taxi for confirmed requests.
\item [{{[}G3{]}}] Allow a registered passenger to have a personal area.
\item [{{[}G4{]}}] Allow a registered passenger to reserve a taxi.
\item [{{[}G5{]}}] Allow a registered passenger to cancel or modify a previous
reservation. 
\item [{{[}G6{]}}] Allow a taxi driver to either accept or reject a request
coming from the system.
\item [{{[}G7{]}}] Allow a taxi driver to inform the system about his/her
availability.
\item [{{[}G8{]}}] Ensure that available taxi queues enjoy the properties
specified in sub paragraph 1.6.2.
\end{lyxlist}

\subsubsection{Queue management}

This paragraph is aimed to give a more precise definition of ``fair
management'' of the available taxi queues.

The city is divided into several zones, to each zone a taxi queue
is assigned. Each zone is characterized by a different load of requests
$n_{i}$ measured in request/minute. Let $N$ be the total number
of taxis available at a certain moment, the number $q_{i}$ of taxis
available in the zone $i$ has to be proportional to $n_{i}$, in
particular $q_{i}=Nn_{i}/\sum_{i}n_{i}$. Every time one taxi turns
from available to busy or out of service or viceversa a new distribution
of the taxis has to be computed; the taxis to be moved have to minimize
the cost of movement calculated as number of zones passed through.
To prevent too many movements, a fluctuation between -30\% and 30\%
from the value $q_{i}$ is accepted without performing taxi movements.


\subsection{Reference documents }
\begin{lyxlist}{00.00.0000}
\item [{{[}1{]}}] IEEE Software Engineering Standards Committee, “IEEE
Std 830-1998, IEEE Recommended Practice for Software Requirements
Specifications”, October 20, 1998. 
\item [{{[}2{]}}] P, Zave, M. Jackson, Four dark corners of requirements
engineering, TOSEM 1997.
\item [{{[}3{]}}] Software Abstractions: Logic, Language, and Analysis,
revised edition Edition by Daniel Jackson, MIT Press.
\item [{{[}4{]}}] Software Engineering 2 course slides.
\end{lyxlist}

\subsection{Overview}

This document is drawn up in accordance to the IEEE Std 830-1998 for
Software Requirements Specifications and it is composed of four sections
and an appendix.
\begin{itemize}
\item The first, this one, section gives a general description of the document
and brief information about the actors and the purposes of the software.
\item The second section provides an overview of the software, highlighting
the interaction with external system interfaces and explaining the
main functions carried out. It also focuses on constraints and domain
assumptions.
\item The third section is entirely dedicated to the derivation and specification
of the requirements. Several scenarios expressed in natural language
will be provided. A generalization of the set of scenarios will be
specified as a set of use cases that will be expressed both in natural
language and using UML use case diagram. For some groups of use cases
a dynamical description will be provided mainly using UML sequence
diagram. A high level conceptual description of the classes affected
by the system will be given using UML class diagram. For some of the
objects involved, we will design a UML state chart diagram showing
the evolution of its state. 
\item The fourth section presents a formalization of a subset of the requirements
using Alloy; some significant instances will be shown.
\item The appendix contains an interesting model of the goals, a brief description
of the tools used to produce this documents and the number of hours
each group member has worked towards the fulfillment of this deadline.\end{itemize}

