
\subsection{Purpose}

\emph{Project management} represents a necessary condition for the
success of any software project. Considered the intrinsic complexity
of a software project, the difficulty in assessing the quality of
the software and the organizational, economical, social and technical
issues to be tackled in any enterprise environment, project management
cannot be avoided. It comprises all the activities aimed to ensuring
the delivering of the software on schedule and in accordance with
context and requirements; in particular project planning, reporting,
risk management and people management. The \emph{PP} (\emph{Project
Planning}) is intended to be a trace of the myTaxiService project
process. For academic reasons, this document is delivered as last
assignment but it refers to both activities to be completed before
the project initiation and activities to be performed during and at
the end of the project. In particular this document is focused on
\emph{project planning} (estimation and scheduling of the process
development, assignment of resources),\emph{ risk management }(definitions,
strategies to tackle risk) and \emph{cost estimation}. In an enterprise
environment the first two are typically performed before the project
starts while the latter is executed either after the requirement engineering
or after the design phase.

This document is intended to be read by stakeholders in order to show
the devised organization of resources and time and to have a general
overview of the effort needed to carry out the project useful for
the evaluation of the founding.


\subsection{Scope}

The \emph{myTaxyService} is an application intended to optimize taxi
service in a large city, making the access to service simpler for
the passengers and ensuring a fair management of the taxi queues. 

Passengers will be able to request a taxi either through a web application
or a mobile app; of course the ``traditional'' ways to call for
a taxi, like a phone call or stopping the taxi along the road, will
be still available and integrated into the system to-be. The software
will make the procedure of calling a taxi simpler (by using GPS information
passenger doesn't need to know the address if the taxi is needed for
the current position) and more usable (passenger will be provided
with information about the waiting time). Moreover, by means of the
application, the passenger can reserve a taxi for a certain date and
time, specifying the origin and the destination of the ride.

Taxi drivers will use a mobile app to inform the system about their
availability and to confirm that they are going to take care of a
call (or to reject it for any reason). The software will make the
taxi management more efficient: the system will be able to identify
the position of each taxi by using GPS; the city will be divided in
virtual zones and a suitable distribution of the taxi among the zones
will automatically be computed.


\subsection{Definitions, Acronyms, Abbreviations}

In this paragraph all the terms, acronyms and abbreviations used in
the following sections are listed.


\subsubsection{Definitions}
\begin{itemize}
\item \emph{Request}: the action performed by the passenger of calling a
taxi for the current position.
\item \emph{Confirmed request}: a request that has been accepted by a taxi
driver.
\item \emph{Reservation}: the action performed by the passenger of booking
a taxi for a specific address and specific date and time.
\item \emph{Waiting time}: an estimation of the time required to taxi driver
to get to passenger's position.
\item \emph{Taxi code}: a unique alphanumerical identifier of the taxi.
\item \emph{Available taxi queues}: data structures used to store the references
of the available taxis, also used to select the taxis to which forward
a request.
\item \emph{Automatic geolocalization}: a system that provides the geographic
coordinates of the user. For this document it can be either a GPS
system or browser geolocalization.
\item \emph{Passengers' application}: the applications used by passengers
to access to TS system. For this document it can be either PMA or
PWA.
\item \emph{Login credentials}: username and password.
\item \textit{Notification}: communication from TS to taxi driver to move
to a specific zone.
\end{itemize}

\subsubsection{Acronyms}
\begin{itemize}
\item TS: myTaxiService.
\item PMA: Passenger mobile application.
\item PWA: Passenger web application.
\item TMA: Taxi driver mobile application.
\item SLOC: Source Lines Of Code.
\item FP(s): Function Point(s).
\end{itemize}
Other acronyms are explained in the corresponding sections.


\subsection{Reference documents}
\begin{lyxlist}{00.00.0000}
\item [{{[}1{]}}] The assignment of the \emph{myTaxiService}.
\item [{{[}2{]}}] RASD (Requirements Analysis and Specification Document)
of the \emph{myTaxiService}.
\item [{{[}3{]}}] DD (Design Document) of the \emph{myTaxiService}.
\item [{{[}4{]}}] ITPD (Integration Testing Plan Document) of the \emph{myTaxiService}.
\item [{{[}5{]}}] Software Engineering 2 course slides.
\item [{{[}6{]}}] COCOMO II Model Definition Manual \url{http://csse.usc.edu/csse/research/COCOMOII/cocomo2000.0/CII_modelman2000.0.pdf}
\item [{{[}7{]}}] Function Point Languages Table Version 5.0 \url{http://www.qsm.com/resources/ function-point-languages-table}
\end{lyxlist}

\subsection{Document Structure}

This document is composed of six sections and an appendix.
\begin{itemize}
\item The first section, this one, is intended to define the goal of the
document, a very high level description of the main functionalists
of the \emph{myTaxiService} system and the resources used to draw
up this document.
\item The second section describes some preliminary results for cost estimation.
A brief theoretical introduction about Function Points method will
be provided and then it will be applied to the myTaxiService system
in order to estimate the complexity of the project and derive the
expected number of SLOC.
\item The third section is devoted to cost estimation. A general method,
called COCOMO, will be presented and applied to the specific case
of myTaxiService in order to estimate the effort of the project, the
estimated time needed for the fulfillment and the number of people
required.
\item In the fourth section we discuss project planning and in particular,
according to the activities identified we propose a possible schedule,
according to both the real deadlines and some reasonable considerations.
We also provide a graphical representation of the schedule by means
of a Gantt chart. 
\item The fifth section is still devoted to project planning but the focus
is on the resource allocation. Some general consideration will be
given and the allocation strategy explained, we will also use a resource
chart to clarify the results.
\item The sixth section is devoted to risk management. The main risks affecting
the myTaxiService project will be stated according to the traditional
classification. Some consideration about the strategy adopted to tackle
them will be given.
\item The appendix contains a brief description of the tools used to produce
this documents and the number of hours each group member has worked
towards the fulfillment of this deadline and the revision history.\end{itemize}

